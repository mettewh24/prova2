\documentclass{article}
\usepackage[utf8]{inputenc}
\usepackage[margin=1in]{geometry}
\usepackage{float}

\title{Misura della caratteristica I-V di un transistor BJT}
\author{Matteo Bonazzi}


\begin{document}

\maketitle
\begin{abstract}
Misura della caratteristica I-V di un transistor BJT in configurazione emettitore comune, in due valori differenti della corrente di base \\
Dal fit lineare dei dati nella regione attiva, si ottengono i parametri $V_{Early}=$ e $R=$.
\end{abstract}

\section{Introduzione}

\section{Dati}

\begin{table}[H]
    \centering
    \begin{tabular}{|c|c|c|c|}
    \hline
    $V_{ce}$ (mV) & Errore V & Risoluzione (mV) & Fondo scala (mV/div) \\ \hline
    4000          & 160      & 200              & 1000                 \\ \hline
    3800          & 150      & 200              & 1000                 \\ \hline
    3600          & 150      & 200              & 1000                 \\ \hline
    3400          & 143      & 200              & 1000                 \\ \hline
    3200          & 139      & 200              & 1000                 \\ \hline
    3000          & 135      & 200              & 1000                 \\ \hline
    2900          & 100      & 200              & 1000                 \\ \hline
    2700          & 95       & 200              & 1000                 \\ \hline
    2500          & 90       & 100              & 500                  \\ \hline
    2400          & 88       & 100              & 500                  \\ \hline
    2200          & 83       & 100              & 500                  \\ \hline
    2000          & 78       & 100              & 500                  \\ \hline
    1900          & 76       & 100              & 500                  \\ \hline
    1700          & 71       & 100              & 500                  \\ \hline
    1500          & 67       & 100              & 500                  \\ \hline
    1400          & 65       & 100              & 500                  \\ \hline
    1200          & 41       & 40               & 200                  \\ \hline
    1120          & 39       & 40               & 200                  \\ \hline
    1000          & 36       & 40               & 200                  \\ \hline
    800           & 31       & 40               & 200                  \\ \hline
    720           & 29       & 40               & 200                  \\ \hline
    500           & 18       & 20               & 100                  \\ \hline
    400           & 16       & 20               & 100                  \\ \hline
    300           & 10       & 10               & 50                   \\ \hline
    200           & 7.8      & 10               & 50                   \\ \hline
    50            & 5.2      & 10               & 50                   \\ \hline
    \end{tabular}
    \end{table}

\begin{table}[H]
    \centering
    \begin{tabular}{|c|c|c|c|}
    \hline
    $I_c$ & errore $I_c$ & Risoluzione (mA) & Fondo scala (mA) \\ \hline
    36.9  & 0.18         & 0.1              & 200              \\ \hline
    36.5  & 0.18         & 0.1              & 200              \\ \hline
    36    & 0.18         & 0.1              & 200              \\ \hline
    35.6  & 0.18         & 0.1              & 200              \\ \hline
    35.1  & 0.18         & 0.1              & 200              \\ \hline
    34.7  & 0.17         & 0.1              & 200              \\ \hline
    34.6  & 0.17         & 0.1              & 200              \\ \hline
    34.2  & 0.17         & 0.1              & 200              \\ \hline
    33.6  & 0.17         & 0.1              & 200              \\ \hline
    33.6  & 0.17         & 0.1              & 200              \\ \hline
    33.1  & 0.17         & 0.1              & 200              \\ \hline
    32.5  & 0.16         & 0.1              & 200              \\ \hline
    32.5  & 0.16         & 0.1              & 200              \\ \hline
    32    & 0.16         & 0.1              & 200              \\ \hline
    31.4  & 0.16         & 0.1              & 200              \\ \hline
    31.2  & 0.16         & 0.1              & 200              \\ \hline
    30.8  & 0.15         & 0.1              & 200              \\ \hline
    30.6  & 0.15         & 0.1              & 200              \\ \hline
    30.2  & 0.15         & 0.1              & 200              \\ \hline
    29.8  & 0.15         & 0.1              & 200              \\ \hline
    28.9  & 0.14         & 0.1              & 200              \\ \hline
    26.5  & 0.13         & 0.1              & 200              \\ \hline
    24.4  & 0.12         & 0.1              & 200              \\ \hline
    22    & 0.11         & 0.1              & 200              \\ \hline
    17.08 & 0.085        & 0.01             & 20               \\ \hline
    4.5   & 0.0225       & 0.01             & 20               \\ \hline
    \end{tabular}
\end{table}


\section{Analisi dati}

\end{document}